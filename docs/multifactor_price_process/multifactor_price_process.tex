\documentclass{article}
\usepackage{amsmath}
\usepackage{mathrsfs}
\usepackage{amssymb} % for "\mathbb" macro
\usepackage[round]{natbib}
\usepackage{url}
\usepackage{hyperref}
\usepackage[toc,page]{appendix}

\title{Multi-Factor Commodity Price Process: Spot and Forward Price Simulation}
\author{Jake C. Fowler}
\date{January 2020}

\begin{document}
\newcommand{\+}[1]{\ensuremath{\mathbf{#1}}}

\maketitle

WARNING: THIS DOCUMENT IS CURRENTLY WORK IN PROGRESS

\tableofcontents

\section{Introduction}
This document describes a very general multi-factor commodity price process model,
together with some analytical results which can be used to explore the assumed
price dynamics, and simulate prices computationally when performing a Monte-Carlo
simulation.

\subsection{Model SDE}

\begin{align}
    \label{eq:forward_sde}
    \frac{dF(t, T)}{F(t, T)}=\sum_{i=1}^n \sigma_i(T)e^{-\alpha_i(T-t)}dz_i(t) \\
    \nonumber
    \alpha_i \in \mathbb{R}_{\ge 0} \\
    \nonumber
    t \in \mathbb{R}_{\ge 0} \\
    \nonumber
    T \in \{ T_0, T_1, T_2, \hdots | T_j \ge t  \} \\
    \nonumber
    \sigma_i : \mathbb{R}_{\ge 0} \rightarrow \mathbb{R}
\end{align}


Where $z_i(t)$ follow correlated Wiener processes with correlation $\rho_{i, j}$, i.e.

\begin{equation}
    \mathbb{E}[dz_i(t)dz_j(t)] = \rho_{i, j}dt
\end{equation}

$F(t, T)$ is the forward price observed at time $t$, for delivery over the period
starting at time $T$.

\subsection{Comparison With Other Models}

This model is almost a specific case of the general multi-factor model presented
in \cite{Clewlow}, which instead of a coefficient of the Brownian motion of
$\sigma_i(T)e^{-\alpha_i(T-t)}$ presented here, uses the more general form
$\sigma_i(t, T)$. As results presented below show, the advantage of the form
used in this paper is that the future distribution of forward and spot prices
can be derived analyticallly. Another difference is that the cited paper assumes the driving stochastic factors
are uncorrelated Brownian motions, whereas the model presented here uses correlated
Brownian motions.

\bigskip

The model is also similar to that presented in \cite{Warin} except that the $\sigma_i$
volatility in this model is a function of forward maturity date $T$, rather than
ovservation date $t$ in \cite{Warin}.

\subsection{Analysis of Model}
The following observations can be made about this model:

\begin{itemize}
    \item Forward prices are lognormally distributed, which has the following practical
    considerations: 
    \begin{itemize}
        \item The model doesn't take into account volatility smile. This assumption is
        fine for the situation of there not being a liquid market in vanilla 
        options and/or the model is being used for a complex product which isn't sensitive
        to changes in the shape of the skew surface. For example this model probably isn't
        suitable for pricing Asian options on Brent crude oil as the vanilla options market
        is liquid enough to be able to extract a volatility surface (i.e. implied volatility
        by expiry and maturity) in order to calibrate a model to take the surface into account
        (e.g. a stochastic volatility model) and price a relativeley simple product like an 
        Asian option. Potentially this model is suitable for the price dynamics part of a 
        more complex model for 
        optimising a physical oil asset such as an oil refinery, where the 
        practical implementation of the optimisation algorithm is only possible with a relatively
        simple deterministic volatility model such as the one presented here. A similar reasoning
        could be used to jusfify the suitability for use for pricing gas storage, swing, and
        power generation assets, as these require complex optimisation. But the reason would
        be even more compelling for such assets as, in many Europe markets, vanilla options
        markets either do not exist, or are not liquid enough to reliably calibrate a
        stochastic volatility model.
        \item The price of some commodities can occasionally go negative, for example
        power and natural gas. Whether this model is suitable depends on the frequency
        of such negative price occurances, and whether the use of the model is impacted
        by this. For example it definetely shouldn't be used to price half-hourly exercise
        European options on the spot power with strike price $\leq 0$.
    \end{itemize}
    \item The model is fitted to the forward curve, i.e. the expected spot price 
    is equal to the equivalent point on the initial curve. This makes the model suitable
    for use under the risk-neutral probability measure assuming deterministic interest
    rates.
    \item There are $n$ sources of uncertainty for which the effect on $F(t, T)$
    depends on the following parameters:
    \begin{itemize}
        \item The impact of each factor's movement on forward prices decays exponentially
        with time to maturity (i.e. $T-t$) with the $\alpha_i$ parameter determining the
        rate of this decay. A higher $\alpha_i$ results in a more rapid decay, and the minimum
        possible value of zero for $\alpha_i$ means that movements in the $i^\text{th}$ 
        stochastic factor result in 
        non-decaying parrallel shifts in the whole forward curve, assuming $\sigma_i(T)$ 
        returns a constant value. As the derivation of the spot price process shows
        below, in terms of spot price dynamics $\alpha_i$ translates to the mean reversion 
        rate of the $i^\text{th}$ stochastic factor driving the spot price.
        \item $\sigma_i$ is an arbitray maturity dependent volatility function. This parameter
        can be used to control seasonality in the volatility and correlation of the forward
        prices. For example a $\sigma_i$ function which only returns non-zero values for $T$
        which are in the Winter delivery periods would make the $i^\text{th}$ factor only
        drive Winter forward prices. $\sigma_i$ could also be used during parameter calibration
        to impose a scaling factor such that the model exactly matches at-the-money implied
        volatilities. A similar calibration step is described in \cite{Rebonato}, p. 673, 
        although in the context of an LMM-style model.
    \end{itemize}
\end{itemize}

\section{Analytical Result on Distribution of Prices}
\subsection{Forward Price Distrubtion}
Integrating this SDE (see Appendix \ref{appendix:forward_sde_integrate}):

\begin{equation}
    \label{eq:forward_process_integrated}
    F(t_2, T) = F(t_1, t)e^{- \frac{1}{2} V(t_1, t_2, T) + I(t_1, t_2, T)}
\end{equation}

\subsubsection{Forward Covariance Surface}
A closed form expression exists for the integrated covariances of
forward price log returns. This is derived in Appendix \ref{appendix:fwd_covar}. The evaluation of such 
covariances is essential to
understand the statistical properties that the model implies for the dynamics of 
the forward curve. It also allows the calibration of model parameters to option
market implied volatilities and historical covariances. 
Defining $C(t_1, t_2, T_1, T_2)$
as the integrated covariance from $t_1$ to $t_2$, between the log returns of forward 
contracts delivering on respective periods starting on $T_1$ and $T_2$.

\begin{equation}
    C(t_1, t_2, T_1, T_2) = \sum_{i=1}^n \sigma_i(T) \sum_{j=1}^n \sigma_j(T) \rho_{i,j} 
    e^{-\alpha_i T_1 -\alpha_j T_2} \vartheta_{-t_1, -t_2}(\alpha_i + \alpha_j)
\end{equation}

\appendix
\appendixpage
\section{Integration of Forward Price SDE}
\label{appendix:forward_sde_integrate}
This appendix gives the detailed step for integrated the forward price process
SDE, hence getting from \ref{eq:forward_sde} to \ref{eq:forward_process_integrated}.

\bigskip

Using Ito's Lemma to calculate the stochastic differential of the natural logarithm
of the forward price.

\begin{equation}
    \label{eq:ln_forward_sde}
    d\ln(F(t, T)) = \frac{1}{F(t, T)} dF(t, T) - \frac{1}{2}
        \frac{1}{F(t, T)^2} (dF(t, T))^2
\end{equation}

Using the properties of Brownian Motion $(dz(t))^2=dt$ and $dtdz(t)=0$ the forward
price differential squared can be evaluated as follows.

\begin{equation}
    \label{eq:df_squared}
    (dF(t, T))^2 = F(t, T)^2 \sum_{i=1}^n \sum_{j=1}^n \sigma_i(T) \sigma_j(T)
    e^{-(\alpha_i + \alpha_j)(T-t)}\rho_{i,j}dt
\end{equation}

Substituting for $(dF(t, T))^2$ and $dF(t, T)$ in \ref{eq:ln_forward_sde}.

\begin{equation}
    \label{eq:ln_forward_sde_2}
    d\ln(F(t, T)) = \sum_{i=1}^n \sigma_i(T)e^{-\alpha_i(T-t)}dz_i(t) - 
    \frac{1}{2} \sum_{i=1}^n \sum_{j=1}^n \sigma_i(T) \sigma_j(T)
    e^{-(\alpha_i + \alpha_j)(T-t)}\rho_{i,j}dt
\end{equation}

The purpose of finding the stochastic differential of $ln(F(t, T))$ is to remove the
$F(t, T)$ coefficient of the Brownian motions, leaving an SDE which can be integrated
to a form where the Ito Integrals have non-stochastic integrands, and hence have a normal
distribution with known mean and variance. Integrating \ref{eq:ln_forward_sde_2}:

\begin{equation}
    \ln(F(t_2, T)) = \ln(F(t_1, t)) - \frac{1}{2} V(t_1, t_2, T) + I(t_1, t_2, T)
\end{equation}

Where:

\begin{eqnarray}
    \label{eq:forward_drift_adjust}
    \nonumber
    V(t_1, t_2, T) =  \sum_{i=1}^n \sigma_i(T) \sum_{j=1}^n 
    \sigma_j(T) \rho_{i,j} \int_{t_1}^{t_2} e^{-(\alpha_i + \alpha_j)(T-u)} du\\
    = \sum_{i=1}^n \sigma_i(T) \sum_{j=1}^n \sigma_j(T) \rho_{i,j} 
    \frac{1}{\alpha_i + \alpha_j} (e^{-(\alpha_i + \alpha_j)(T-t_2)} - 
    e^{-(\alpha_i + \alpha_j)(T-t_1)}) \\
    \nonumber
\end{eqnarray}

And:

\begin{equation}
    I(t_1, t_2, T) = \sum_{i=1}^n \sigma_i(T) \int_{t_1}^{t_2} e^{-\alpha_i(T-u)}dz_i(u)
\end{equation}

\bigskip
There is a problem in that $V(t_1, t_2, T)$ is not defined for the case where 
$\alpha_i + \alpha_j = 0$. A continuous extension is required in order for this
case to be allowed. Consider the function $\psi$ which is only defined for 
the domain $\mathbb{R}_{\ne 0}$, the set real numbers, excluding zero.

\begin{equation}
    \psi_{c_1, c_2}(x) = \frac{e^{-x c_2} - e^{-x c_1}}{x} \quad  x \in \mathbb{R}_{\ne 0}
\end{equation}

The first step to defining the continous extension is to write $\psi$ as a Taylor Series
about zero.

\begin{equation}
    \psi_{c_1, c_2}(x) = \frac{1 - x c_2 - 1 + x c_1 + O(x^2)}{x}
\end{equation}

Taking the limit of $\psi(x)$ towards zero, the $O(x^2)$ becomes negligible, hence:

\begin{eqnarray}
    \nonumber
    \lim_{x \to 0} \psi_{c_1, c_2}(x) = \frac{1 - x c_2 - 1 + x c_1}{x} \\
    = c_1 - c_2
\end{eqnarray}

Using this, we can define $\vartheta$ as the continous extension of $\psi$:

\begin{align}
    \vartheta_{c_1, c_2}(x) = 
    \begin{cases}
        \psi_{c_1, c_2}(x) & \text{for} \quad x \in \mathbb{R}_{\ne 0} \\
        c_1 - c_2 & \text{for} \quad x=0
    \end{cases}
\end{align}

Which in the case of equation for V is:

\begin{align}
    \vartheta_{T-t_1, T-t_2}(x) = 
    \begin{cases}
        \psi_{T-t_1, T-t_2}(x) & \text{for} \quad x \in \mathbb{R}_{\ne 0} \\
        t_2 - t_1 & \text{for} \quad x=0
    \end{cases}
\end{align}


\bigskip

Using this, we can redefine $V(t_1, t_2, T)$ to include the case of $\alpha_i$ and
$\alpha_j$ both being equal to zero:

\begin{equation}
    \nonumber
    V(t_1, t_2, T) = \sum_{i=1}^n \sigma_i(T) \sum_{j=1}^n \sigma_j(T) \rho_{i,j} 
    \vartheta_{T-t_1, T-t_2}(\alpha_i + \alpha_j)
\end{equation}

Hence the forward price can be expressed as:


\begin{equation}
    F(t_2, T) = F(t_1, t)e^{- \frac{1}{2} V(t_1, t_2, T) + I(t_1, t_2, T)}
\end{equation}

\section{Integrated Forward Covariances}
\label{appendix:fwd_covar}
This section derives a closed form expression for the integrated covariances of
forward price log returns, a quantity is referred to as "terminal covariance" in
\cite{Rebonato}. Defining $C(t_1, t_2, T_1, T_2)$
as the integrated covariance from $t_1$ to $t_2$, between the log returns of forward 
contracts delivering on respective periods starting on $T_1$ and $T_2$.

\begin{equation}
    C(t_1, t_2, T_1, T_2) = \mathbb{E}\biggl[\ln\biggl(\frac{F(t_2, T_1)}{F(t_1, T_1)} \biggr) 
    \ln\biggl(\frac{F(t_2, T_2)}{F(t_1, T_2)} \biggr) \biggr] \\
\end{equation}

Removing the products of stochastic and deterministic terms, as they will be equal to zero:

\begin{equation}
    C(t_1, t_2, T_1, T_2) = \mathbb{E}\bigl[ I(t_1, t_2, T_1) I(t_1, t_2, T_2) \bigr] 
\end{equation}

Using the known properties of Ito Integrals:
% TODO put in equation with expection of Ito product inside summation?

\begin{eqnarray}
    \nonumber
    C(t_1, t_2, T_1, T_2) = \sum_{i=1}^n \sigma_i(T) \sum_{j=1}^n 
    \sigma_j(T) \rho_{i,j} e^{-\alpha_i T_1 -\alpha_j T_2} 
    \int_{t_1}^{t_2} e^{u(\alpha_i + \alpha_j)} du \\
    = \sum_{i=1}^n \sigma_i(T) \sum_{j=1}^n \sigma_j(T) \rho_{i,j} 
    e^{-\alpha_i T_1 -\alpha_j T_2} \frac{1}{\alpha_i + \alpha_j}
    (e^{t_2(\alpha_i + \alpha_j)} - e^{t_1(\alpha_i + \alpha_j)} )
\end{eqnarray}

Allowing for the case of $\alpha_i$ and $\alpha_j$ both being equal to zero, 
as in the previous appendix, a continuous extension is made to redefine the function
$C$:

\begin{equation}
    C(t_1, t_2, T_1, T_2) = \sum_{i=1}^n \sigma_i(T) \sum_{j=1}^n \sigma_j(T) \rho_{i,j} 
    e^{-\alpha_i T_1 -\alpha_j T_2} \vartheta_{-t_1, -t_2}(\alpha_i + \alpha_j)
\end{equation}


\section{Spot Price Process Distribution and Simulation}
\label{appendix:spot_distribution}
Defining the spot price $S(t)$ as the price for delivery of the commodity in the period
starting at time $t$, observed at time $t$. Using the equivalence between $S(t)$
and $F(t, t)$ and equation \ref{eq:forward_process_integrated}:

\begin{equation}
    S(t) = F(0, t)e^{- \frac{1}{2} V(0, t, t) + I(0, t, t)}
\end{equation}

When simulating a spot price path it is necessary to know the relationship between
$S(t_k)$ and $S(t_{k-1})$, where $0 \leq t_{k-1} < t_k$. Defining $V_s(t)=V(0, t, t)$ and 
$I_s(t)=I(0, t, t)$:

\begin{equation}
    S(t_k) = S(t_{k-1}) \frac{F(0, t_k)}{F(0, t_{k-1})} e^{- \frac{1}{2} (V_s(t_k) - 
    V_s(t_{k-1})) + I_s(t_k) - I_s(t_{k-1})}
\end{equation}

Focussing on the stochastic term $I_s(t_k) - I_s(t_{k-1})$:

\begin{equation}
    I_s(t_k) - I_s(t_{k-1}) = \sum_{i=1}^n \biggl( \sigma_i(t_k) 
    \int_0^{t_k} e^{-\alpha_i(t_k-u)} dz_i(u) - \sigma_i(t_{k-1}) 
    \int_0^{t_{k-1}} e^{-\alpha_i(t_{k-1}-u)} dz_i(u) \biggr)
\end{equation}

Defining:

\begin{equation}
    F_i(t) = \int_0^{t} e^{-\alpha_i(t-u)} dz_i(u)
\end{equation}

Substitution this into the above equation:

\begin{equation}
    I_s(t_k) - I_s(t_{k-1}) = \sum_{i=1}^n \biggl( \sigma_i(t_k) 
    F_i(t_k) - \sigma_i(t_{k-1}) F_i(t_{k-1}) \biggr)
\end{equation}


When simulating, this expression will be adaped to the filtration $\mathcal{F}_{t_{k-1}}$, hence the
$F_i(t_{k-1})$ will have been realised. $F_i(t_k)$ will be part realised, hence it is instructive
to split this into it's deterministic and stochastic parts, as of $t_{k-1}$. First making the
substituting $t_k = t_{k-1} + t_k - t_{k-1}$:

\begin{eqnarray}
    \nonumber
    F_i(t_k) = e^{-\alpha_i(t_k - t_{k-1})} \int_0^{t_k} e^{-\alpha_i(t_{k-1}-u)} dz_i(u) \\
        = e^{-\alpha_i(t_k - t_{k-1})} \biggl(\int_0^{t_{k-1}} e^{-\alpha_i(t_{k-1}-u)} dz_i(u) +
        \int_{t_{k-1}}^{t_k} e^{-\alpha_i(t_{k-1}-u)} dz_i(u) \biggr)
\end{eqnarray}

Noticing that the first integral is the definition of $F_i(t_{k-1})$, the function $F_i$ can
be defined recursively:

\begin{equation}
    F_i(t_k) = e^{-\alpha_i(t_k - t_{k-1})}F_i(t_{k-1}) + \int_{t_{k-1}}^{t_k} e^{-\alpha_i(t_k-u)} dz_i(u)
\end{equation}

For any $t_k > t_{k-1}$ and base case:

\begin{equation}
    F_i(0) = 0
\end{equation}

In this form $F_i(t)$ can be recognised as an Ornstein–Uhlenbeck process with zero drift,
mean reversion parameter $\alpha_i$, and $\sigma$ (coeffient of Brownian motion) of 1.

\bigskip

When simulating the spot price for an increasing sequence of $n$
times $t_{k-1}, t_2 \hdots t_{n-1}, t_n$ where $0 < t_{k-1} < t_k$, for each of these
times the first step is to calculate $F_i(t_k)$ recursively, using the above relationship % TODO word better (maybe equation ref)
for $i = 1, \hdots, n$.

\begin{equation}
    F_i(t_k) = e^{-\alpha_i(t_k - t_{k-1})}F_i(t_{k-1}) + Z_i(t_k)
\end{equation}

Where, using the property of Ito Integrals $Z_i(t_k)$ is multivariate normally 
distributed, with mean zero and covariance:

\begin{equation}
    cov(Z_i, Z_j) = \frac{\rho_{i, j}}{\alpha_i + \alpha_j}(1 - 
    e^{-(\alpha_i + \alpha_j)(t_k - t_{k-1})})
\end{equation}

The spot price can then be calculated from the values of $F_i(t_k)$:

\begin{equation}
    S(t_k) = S(t_{k-1}) \frac{F(0, t_k)}{F(0, t_{k-1})} e^{- \frac{1}{2} (V_s(t_k) - 
    V_s(t_{k-1})) + \sum_{i=1}^n \bigl(\sigma_i(t_k)F_i(t_k) - \sigma_i(t_{k-1})F_i(t_{k-1})\bigr)}
\end{equation}


% TODO significance of F as factor for Markovian
% TODO make multi-commodity

\bibliographystyle{plainnat}
\bibliography{multifactor_price_process}

\end{document}